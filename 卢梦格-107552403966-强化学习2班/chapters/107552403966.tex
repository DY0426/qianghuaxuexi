

\section{探索与利用}
\subsection{基本概念}
\begin{itemize}
    \item \textbf{利用(Exploitation)}:执行已知最优策略以获取稳定收益。
    \item \textbf{探索(Exploration)}:尝试新策略以发现潜在更高收益,可能牺牲短期回报。
    \item \textbf{核心矛盾}:短期收益最大化与长期策略优化的权衡。
\end{itemize}

\subsection{多臂老虎机问题}
\begin{itemize}
    \item \textbf{形式化描述}:
        \begin{itemize}
            \item 动作集合:\( \mathcal{A} = \{a^1, \dots, a^K\} \)
            \item 收益分布:\( \mathcal{R}(r \mid a^t) = \mathbb{P}(r \mid a^t) \)
            \item 目标:最大化累积收益 \( \max \sum_{t=1}^T r_t \)
        \end{itemize}
    \item \textbf{核心挑战}:在未知收益分布下平衡探索与利用。
\end{itemize}

\subsection{经典方法}
\subsubsection{贪心策略与\(\epsilon\)-贪心}
\begin{itemize}
    \item \textbf{贪心策略}:
        \[
        a^* = \arg\max_{a^i} Q(a^i), \quad \text{总遗憾(Regret)线性增长}
        \]
    \item \textbf{\(\epsilon\)-贪心}:
        \[
        a_t = 
        \begin{cases}
            \arg\max_a Q(a) & \text{概率 } 1-\epsilon \\
            \text{随机动作} & \text{概率 } \epsilon
        \end{cases}
        \]
        \begin{itemize}
            \item 总遗憾仍线性增长,但斜率降低。
        \end{itemize}
\end{itemize}

\subsubsection{衰减\(\epsilon\)-贪心}
\begin{itemize}
    \item 随时间衰减探索率:\( \epsilon_t = \min\left\{1, \frac{c|\mathcal{A}|}{d^2 t}\right\} \)
    \item 理论保证:对数渐进遗憾 \( O(\log T) \)。
\end{itemize}

\subsubsection{乐观初始化}
\begin{itemize}
    \item 初始化高估 \( Q(a^i) \),通过增量更新修正:
        \[
        Q(a^i) \leftarrow Q(a^i) + \frac{1}{N(a^i)} (r_t - Q(a^i))
        \]
    \item 缺点:可能陷入局部最优。
\end{itemize}

\subsection{高级方法}
\subsubsection{上置信界(UCB)}
\begin{itemize}
    \item 选择准则:
        \[
        a_t = \arg\max_{a \in \mathcal{A}} \left( \hat{Q}(a) + \sqrt{\frac{\log t}{2 N_t(a)}} \right)
        \]
    \item 理论保证:总遗憾上限 \( \sigma_R \leq 8 \log T \sum_{a|\Delta_a > 0} \Delta_a \)。
\end{itemize}

\subsubsection{汤普森采样(Thompson Sampling)}
\begin{itemize}
    \item 基于贝叶斯后验分布采样:
        \[
        a_t = \arg\max_{a} \theta_a, \quad \theta_a \sim \text{Beta}(S_a + 1, F_a + 1)
        \]
    \item 优点:自动平衡探索与利用,实践表现优异。
\end{itemize}

\subsection{理论下界与总结}
\begin{itemize}
    \item \textbf{Lai \& Robbins下界}:
        \[
        \lim_{T \to \infty} \sigma_R \geq \log T \sum_{a|\Delta_a > 0} \frac{\Delta_a}{D_{KL}(\mathcal{R}_a \| \mathcal{R}^*)}
        \]
    \item \textbf{关键结论}:
        \begin{itemize}
            \item 探索与利用是强化学习的核心问题。
            \item 多臂老虎机是无状态强化学习的简化模型。
            \item 最优方法(如UCB、汤普森采样)可达对数遗憾。
        \end{itemize}
\end{itemize}
\section{马尔可夫决策过程}
\subsection{基础概念}
\begin{itemize}
    \item \textbf{马尔可夫性质}:未来状态仅依赖当前状态,与历史无关:
        \[
        \mathbb{P}[S_{t+1}|S_t] = \mathbb{P}[S_{t+1}|S_1, \dots, S_t]
        \]
    \item \textbf{MDP五元组}:\((S, A, \{P_{sa}\}, \gamma, R)\)
        \begin{itemize}
            \item 状态集 \(S\)、动作集 \(A\)、状态转移概率 \(P_{sa}(s')\)、折扣因子 \(\gamma \in [0,1]\)、奖励函数 \(R(s,a)\)
        \end{itemize}
    \item \textbf{累积奖励}:
        \[
        R(s_0) + \gamma R(s_1) + \gamma^2 R(s_2) + \cdots
        \]
\end{itemize}

\subsection{占用度量(Occupancy Measure)}
\begin{itemize}
    \item 描述策略 \(\pi\) 下状态-动作对的分布:
        \[
        \rho^\pi(s,a) = \mathbb{E}_{a \sim \pi(\cdot|s), s' \sim P(\cdot|s,a)} \left[ \sum_{t=0}^T \gamma^t \mathbb{I}(s_t=s, a_t=a) \right]
        \]
    \item \textbf{关键定理}:
        \begin{itemize}
            \item 占用度量唯一确定策略:\(\rho^{\pi_1} = \rho^{\pi_2} \iff \pi_1 = \pi_2\)
            \item 累积奖励可表示为:\(V(\pi) = \sum_{s,a} \rho^\pi(s,a) R(s,a)\)
        \end{itemize}
\end{itemize}

\subsection{动态规划方法}
\subsubsection{价值函数与Bellman方程}
\begin{itemize}
    \item \textbf{策略价值函数}:
        \[
        V^\pi(s) = \mathbb{E}\left[ \sum_{t=0}^\infty \gamma^t R(s_t) \mid s_0=s, \pi \right]
        \]
    \item \textbf{Bellman方程}:
        \[
        V^\pi(s) = R(s) + \gamma \sum_{s'} P_{s\pi(s)}(s') V^\pi(s')
        \]
    \item \textbf{最优价值函数}:
        \[
        V^*(s) = \max_\pi V^\pi(s) = R(s) + \max_a \gamma \sum_{s'} P_{sa}(s') V^*(s')
        \]
\end{itemize}

\subsubsection{价值迭代与策略迭代}
\begin{itemize}
    \item \textbf{价值迭代}:
        \begin{enumerate}
            \item 初始化 \(V(s) = 0\)
            \item 重复更新直至收敛:
                \[
                V(s) \leftarrow R(s) + \max_a \gamma \sum_{s'} P_{sa}(s') V(s')
                \]
        \end{enumerate}
    \item \textbf{策略迭代}:
        \begin{enumerate}
            \item 随机初始化策略 \(\pi\)
            \item 交替执行:
                \begin{itemize}
                    \item \textbf{策略评估}:计算 \(V^\pi\)(解Bellman方程)
                    \item \textbf{策略改进}:\(\pi(s) \leftarrow \arg\max_a \sum_{s'} P_{sa}(s') V^\pi(s')\)
                \end{itemize}
        \end{enumerate}
    \item \textbf{对比}:
        \begin{itemize}
            \item 价值迭代无显式策略,直接更新价值函数
            \item 策略迭代需显式计算策略价值,收敛更快但计算量大
        \end{itemize}
\end{itemize}

\subsection{基于模型的强化学习}
\begin{itemize}
    \item \textbf{模型学习}:从经验数据估计 \(P_{sa}\) 和 \(R\):
        \[
        P_{sa}(s') = \frac{\text{转移次数}(s,a,s')}{\text{动作次数}(s,a)}, \quad R(s) = \text{平均观测奖励}
        \]
    \item \textbf{算法流程}:
        \begin{enumerate}
            \item 执行策略收集数据
            \item 估计模型参数 \(P_{sa}\) 和 \(R\)
            \item 用动态规划求解新策略
            \item 重复直至收敛
        \end{enumerate}
\end{itemize}

\subsection{总结}
\begin{itemize}
    \item MDP为序列决策问题提供数学框架,核心是状态转移与奖励函数
    \item 动态规划方法(价值/策略迭代)适用于已知模型的白盒环境
    \item 占用度量连接策略与数据分布,是理论分析的重要工具
    \item 黑盒环境中需先学习模型参数,再应用动态规划
\end{itemize}

\section{无模型的强化学习-值函数估计}
\subsection{核心概念}
\begin{itemize}
    \item \textbf{模型无关性}:无需已知MDP模型(状态转移$P_{sa}$和奖励函数$R$),直接从经验片段学习。
    \item \textbf{关键任务}:
        \begin{itemize}
            \item 估计值函数 $V^\pi(s)$ 或 $Q^\pi(s,a)$
            \item 优化策略 $\pi$
        \end{itemize}
\end{itemize}

\subsection{蒙特卡洛方法(MC)}
\begin{itemize}
    \item \textbf{基本思想}:通过完整片段的累计奖励估计值函数:
        \[
        V^\pi(s) \approx \frac{1}{N} \sum_{i=1}^N G_t^{(i)}, \quad G_t = R_{t+1} + \gamma R_{t+2} + \cdots + \gamma^{T-1} R_T
        \]
    \item \textbf{增量更新}:
        \[
        V(S_t) \leftarrow V(S_t) + \alpha \left( G_t - V(S_t) \right)
        \]
    \item \textbf{特性}:
        \begin{itemize}
            \item 无偏估计,但高方差
            \item 需等待片段结束,仅适用于有限片段任务
        \end{itemize}
\end{itemize}

\subsection{时序差分学习(TD)}
\begin{itemize}
    \item \textbf{基本思想}:通过自举(bootstrapping)结合当前奖励与下一状态估计值:
        \[
        V(S_t) \leftarrow V(S_t) + \alpha \left( R_{t+1} + \gamma V(S_{t+1}) - V(S_t) \right)
        \]
    \item \textbf{时序差分误差}:$\delta_t = R_{t+1} + \gamma V(S_{t+1}) - V(S_t)$
    \item \textbf{特性}:
        \begin{itemize}
            \item 低方差,但有偏估计
            \item 可在线学习,适用于连续任务
        \end{itemize}
\end{itemize}

\subsection{离线策略与重要性采样}
\begin{itemize}
    \item \textbf{目标}:利用策略$\mu$生成的样本评估目标策略$\pi$。
    \item \textbf{重要性比率}:
        \[
        \rho_t = \frac{\pi(a_t|s_t)}{\mu(a_t|s_t)}
        \]
    \item \textbf{加权更新}:
        \begin{itemize}
            \item MC离线策略:$G_t^{\pi/\mu} = \prod_{k=t}^T \rho_k G_t$
            \item TD离线策略:$V(s_t) \leftarrow V(s_t) + \alpha \left( \rho_t (r_{t+1} + \gamma V(s_{t+1})) - V(s_t) \right)$
        \end{itemize}
    \item \textbf{局限性}:高方差,需满足$\pi(a|s) > 0 \Rightarrow \mu(a|s) > 0$。
\end{itemize}

\subsection{MC与TD对比}

\begin{table}[H]
\centering
\caption{蒙特卡洛(MC)与时序差分(TD)对比}
\begin{tabular}{lcc}
\toprule
\textbf{特性} & \textbf{蒙特卡洛(MC)} & \textbf{时序差分(TD)} \\
\midrule
偏差 & 无偏 & 有偏 \\
方差 & 高 & 低 \\
在线学习 & 不支持 & 支持 \\
适用范围 & 片段化任务 & 连续/片段化任务 \\
收敛性 & 稳定 & 对初始值敏感 \\
\bottomrule
\end{tabular}
\label{tab:mc_vs_td}
\end{table}

\subsection{总结}
\begin{itemize}
    \item \textbf{MC}:直接估计,高方差,适合小规模离散任务。
    \item \textbf{TD}:高效自举,低方差,适合大规模或连续任务。
    \item \textbf{离线策略}:通过重要性采样复用数据,但需权衡方差与稳定性。
    \item \textbf{核心挑战}:在偏差与方差间取得平衡,选择适合问题特性的方法。
\end{itemize}

\section{Sarsa与Q学习}
\subsection{SARSA算法}
\begin{itemize}
    \item \textbf{在线策略TD控制}:基于当前策略生成的数据进行学习
    \item \textbf{更新公式}:
        \[
        Q(s,a) \leftarrow Q(s,a) + \alpha \left( r + \gamma Q(s',a') - Q(s,a) \right)
        \]
    \item \textbf{算法流程}:
        \begin{enumerate}
            \item 初始化$Q(s,a)$
            \item 使用$\epsilon$-greedy策略选择动作$a$
            \item 执行$a$,观察$(r,s')$
            \item 在$s'$用相同策略选择$a'$
            \item 更新$Q(s,a)$
        \end{enumerate}
    \item \textbf{特性}:
        \begin{itemize}
            \item 保守策略:考虑探索带来的风险(如Windy Gridworld示例)
            \item 适合需要安全探索的场景
        \end{itemize}
\end{itemize}

\subsection{Q学习算法}
\begin{itemize}
    \item \textbf{离线策略TD控制}:可学习不同于行为策略的目标策略
    \item \textbf{更新公式}:
        \[
        Q(s,a) \leftarrow Q(s,a) + \alpha \left( r + \gamma \max_{a'} Q(s',a') - Q(s,a) \right)
        \]
    \item \textbf{关键特点}:
        \begin{itemize}
            \item 直接学习最优策略$Q^*$
            \item 无需重要性采样(因使用max操作)
            \item 行为策略$\mu$与目标策略$\pi$分离
        \end{itemize}
    \item \textbf{收敛性证明}:
        \begin{itemize}
            \item 收缩算子理论:$\|HQ_1-HQ_2\|_\infty \leq \gamma\|Q_1-Q_2\|_\infty$
            \item 柯西收敛准则保证$Q\rightarrow Q^*$
        \end{itemize}
\end{itemize}

\subsection{对比分析}
\begin{table}[H]
\centering
\caption{SARSA与Q学习对比}
\begin{tabular}{lll}  % 改为标准三列左对齐
\toprule
\textbf{特性} & \textbf{SARSA} & \textbf{Q学习} \\
\midrule
策略类型 & 在线策略 & 离线策略 \\
更新目标 & $r+\gamma Q(s',a')$ & $r+\gamma \max\limits_{a'}Q(s',a')$ \\
探索风险 & 保守(考虑策略随机性) & 激进(总假设最优动作) \\
适用场景 & 安全关键任务(如机器人控制) & 最大化收益任务(如游戏AI) \\
收敛性 & 到最优$\epsilon$-soft策略 & 到真正最优策略 \\
\bottomrule
\end{tabular}
\label{tab:sarsa_vs_q}
\end{table}

\subsection{实验示例}
\begin{itemize}
    \item \textbf{Windy Gridworld}:
        \begin{itemize}
            \item SARSA学习规避风险路径
            \item 随训练逐渐优化步数
        \end{itemize}
    \item \textbf{Cliff-walking}:
        \begin{itemize}
            \item SARSA选择安全路径(更长但更可靠)
            \item Q学习选择最优路径(高风险高回报)
        \end{itemize}
\end{itemize}

\subsection{总结}
\begin{itemize}
    \item SARSA适合需要平衡探索安全性的场景
    \item Q学习能直接收敛到最优策略但可能过度乐观
    \item 实际选择取决于任务对风险与收益的权衡
\end{itemize}

\section{多步Sarsa}
\subsection{多步时序差分学习}
\subsubsection{多步TD预测}
\begin{itemize}
    \item \textbf{n步累计奖励}:
        \[
        G_t^{(n)} = R_{t+1} + \gamma R_{t+2} + \cdots + \gamma^{n-1}R_{t+n} + \gamma^n V(S_{t+n})
        \]
    \item \textbf{更新规则}:
        \[
        V(S_t) \leftarrow V(S_t) + \alpha \left( G_t^{(n)} - V(S_t) \right)
        \]
    \item \textbf{特殊情形}:
        \begin{itemize}
            \item $n=1$:TD(0)
            \item $n=\infty$:蒙特卡洛方法
        \end{itemize}
\end{itemize}

\subsubsection{TD($\lambda$)算法}
\begin{itemize}
    \item \textbf{$\lambda$-累计奖励}:
        \[
        G_t^\lambda = (1-\lambda)\sum_{n=1}^{T-t-1}\lambda^{n-1}G_t^{(n)} + \lambda^{T-t-1}G_t
        \]
    \item \textbf{权重分配}:
        \begin{center}
            \begin{table}[H]  % 使用[H]强制定位
                \centering
                \caption{$\lambda$值与等效算法对应关系}
                    \begin{tabular}{cc}
                    \toprule
                    $\lambda$值 & 等效算法 \\
                    \midrule
                    0 & TD(0) \\
                    1 & 蒙特卡洛 \\
                    $(0,1)$ & 多步TD加权平均 \\
                    \bottomrule
                    \end{tabular}
            \end{table}
        \end{center}
\end{itemize}

\subsection{多步SARSA算法}
\subsubsection{基本形式}
\[
G_{t:t+n} = R_{t+1} + \gamma R_{t+2} + \cdots + \gamma^{n-1}R_{t+n} + \gamma^n Q(S_{t+n},A_{t+n})
\]
\[
Q(S_t,A_t) \leftarrow Q(S_t,A_t) + \alpha \left( G_{t:t+n} - Q(S_t,A_t) \right)
\]

\subsubsection{离线策略多步SARSA}
\begin{itemize}
    \item \textbf{重要性采样比率}:
        \[
        \rho_{t:h} = \prod_{k=t}^{\min(h,T-1)} \frac{\pi(A_k|S_k)}{b(A_k|S_k)}
        \]
    \item \textbf{加权更新}:
        \[
        Q(S_t,A_t) \leftarrow Q(S_t,A_t) + \alpha \rho_{t+1:t+n} \left( G_{t:t+n} - Q(S_t,A_t) \right)
        \]
\end{itemize}

\subsection{多步树回溯算法}
\begin{itemize}
    \item \textbf{核心思想}:通过期望值替代采样,避免重要性采样
    \item \textbf{n步更新目标}:
        \[
        G_{t:t+n} = R_{t+1} + \gamma \sum_{a \neq A_{t+1}} \pi(a|S_{t+1})Q(S_{t+1},a) + \gamma \pi(A_{t+1}|S_{t+1})G_{t+1:t+n}
        \]
    \item \textbf{优势}:
        \begin{itemize}
            \item 无重要性采样带来的高方差
            \item 保持离线策略学习的灵活性
        \end{itemize}
\end{itemize}

\subsection{各时序差分控制算法对比}
\begin{table}[H]  % 使用[H]强制定位
\centering
\caption{时序差分控制算法对比}
\begin{tabular}{llll}
\toprule
\textbf{算法} & \textbf{策略类型} & \textbf{更新方式} & \textbf{复杂度} \\
\midrule
单步SARSA & 在线策略 & $Q(s,a) \leftarrow r+\gamma Q(s',a')$ & 低 \\
多步SARSA & 在线策略 & $n$步累计奖励 & 中 \\
Q学习 & 离线策略 & $\max_{a'}Q(s',a')$ & 低 \\
树回溯 & 离线策略 & 期望值展开 & 高 \\
\bottomrule
\end{tabular}
\end{table}

\subsection{实验分析}
\begin{itemize}
    \item \textbf{随机游走任务}:
        \begin{itemize}
            \item 多步TD($\lambda$)在$\lambda \approx 0.9$时取得最佳表现
            \item n=4的多步方法优于单步和蒙特卡洛
        \end{itemize}
    \item \textbf{悬崖漫步任务}:
        \begin{itemize}
            \item 多步SARSA比Q学习更保守
            \item 树回溯算法平衡了探索与利用
        \end{itemize}
\end{itemize}

\subsection{总结}
\begin{itemize}
    \item 多步方法在偏差-方差权衡上优于单步方法
    \item TD($\lambda$)通过参数$\lambda$灵活调整时间跨度
    \item 树回溯算法为离线策略学习提供了新思路
    \item 实际应用需根据任务特性选择时间步长n和$\lambda$
\end{itemize}

\section{规划与学习}
\subsection{规划与学习 (1)}
\subsubsection{规划与学习基础}
\paragraph{策略评估与提升}
\begin{itemize}
    \item \textbf{策略评估}:
        \[
        V^\pi(s) = \mathbb{E}_\pi\left[\sum_{k=0}^\infty \gamma^k R_{t+k+1} \mid S_t=s\right]
        \]
    \item \textbf{策略提升定理}:若$\pi'$满足$Q^\pi(s,\pi'(s)) \geq V^\pi(s)$,则$\pi'$优于$\pi$
        \[
        V^{\pi'}(s) \geq V^\pi(s),\quad \forall s \in \mathcal{S}
        \]
\end{itemize}

\subsubsection{模型与规划}
\begin{itemize}
    \item \textbf{模型定义}:
        \begin{itemize}
            \item 分布模型:$p(s',r|s,a)$
            \item 样本模型:生成$(s',r)$样本
        \end{itemize}
    \item \textbf{规划分类}:
        \begin{table}[H]  % 使用[H]强制定位
            \centering
            \caption{规划方法类型对比}
            \begin{tabular}{ll}
            \toprule
            \textbf{类型} & \textbf{特点} \\
            \midrule
            状态空间规划 & 在状态空间中搜索最优策略 \\
            规划空间规划 & 在动作序列空间中搜索 \\
            \bottomrule
            \end{tabular}
        \end{table}

\end{itemize}

\subsubsection{Dyna架构}
\paragraph{核心思想}
\begin{itemize}
    \item \textbf{三要素集成}:
        \begin{itemize}
            \item 真实经验:与环境交互
            \item 模型学习:估计$p(s',r|s,a)$
            \item 规划:利用模型生成模拟经验
        \end{itemize}
    \item \textbf{数据流}:
        \begin{figure}[h]
\centering
\begin{tikzpicture}[node distance=6cm]
    \node (env) {环境};
    \node (model) [right of=env] {模型};
    \node (value) [above of=model] {值函数/策略};
    \draw[->] (env) -- node[midway,above] {真实经验} (model);
    \draw[->] (model) -- node[midway,right] {模拟经验} (value);
    \draw[->] (value) -- node[midway,left] {动作} (env);
\end{tikzpicture}
\captionof{figure}{强化学习中的环境-模型-策略交互关系} % 专业图题注
\label{fig:rl_interaction}
\end{figure}
\end{itemize}

\subsubsection{Dyna-Q算法}
\begin{algorithm}[H]
\caption{表格型Dyna-Q}
\begin{algorithmic}[1]
\REQUIRE 状态空间 $\mathcal{S}$,动作空间 $\mathcal{A}(s)$
\FOR{每个 $s \in \mathcal{S}$, $a \in \mathcal{A}(s)$}
    \STATE 初始化 $Q(s,a)$ 和 $\text{Model}(s,a)$
\ENDFOR
\WHILE{未收敛}
    \STATE $S \leftarrow$ 当前状态
    \STATE $A \leftarrow \epsilon$-greedy($S,Q$)
    \STATE 执行 $A$,观测 $R,S'$
    \STATE $Q(S,A) \leftarrow Q(S,A) + \alpha[R+\gamma\max_a Q(S',a)-Q(S,A)]$
    \STATE $\text{Model}(S,A) \leftarrow (R,S')$
    \FOR{$i=1$ to $n$}
        \STATE $S \leftarrow$ 随机已访问状态
        \STATE $A \leftarrow$ 随机已执行动作
        \STATE $(R,S') \leftarrow \text{Model}(S,A)$
        \STATE $Q(S,A) \leftarrow Q(S,A) + \alpha[R+\gamma\max_a Q(S',a)-Q(S,A)]$
    \ENDFOR
\ENDWHILE
\end{algorithmic}
\end{algorithm}


\subsubsection{实验与分析}
\begin{itemize}
    \item \textbf{迷宫任务}:
        \begin{itemize}
            \item 规划步数$n$越大,收敛越快
            \item $n=50$时比纯无模型学习快10倍
        \end{itemize}
    \item \textbf{动态环境适应}:
        \begin{itemize}
            \item Dyna-Q+可检测环境变化(如捷径出现)
            \item 通过给未访问状态动作对附加奖励实现
        \end{itemize}
\end{itemize}

\subsubsection{理论启示}
\begin{itemize}
    \item \textbf{规划-学习对偶性}:
        \begin{itemize}
            \item 规划:基于模型的策略搜索
            \item 学习:基于经验的模型构建
        \end{itemize}
    \item \textbf{关键优势}:
        \begin{itemize}
            \item 样本效率提升:复用真实经验
            \item 灵活中断:可随时停止规划过程
        \end{itemize}
\end{itemize}
\subsection{规划与学习 (2)}
\subsubsection{高效规划方法}
\paragraph{优先级采样}
\begin{itemize}
    \item \textbf{核心思想}:根据Bellman误差动态调整更新优先级
    \[
    P = \left| R + \gamma \max_a Q(S',a) - Q(S,A) \right|
    \]
    \item \textbf{算法流程}:
        \begin{enumerate}
            \item 维护优先级队列$PQueue$
            \item 当$P > \theta$时插入队列
            \item 优先更新高误差的状态动作对
        \end{enumerate}
    \item \textbf{优势}:迷宫任务中比Dyna-Q快3-5倍
\end{itemize}

\subsubsection{期望更新 vs 采样更新}
    \begin{table}[H]  % 使用[H]强制定位
        \centering
        \caption{两种更新方式对比}
        \begin{tabular}{lll}
        \toprule
        \textbf{类型} & \textbf{计算复杂度} & \textbf{适用场景} \\
        \midrule
        期望更新 & $O(|\mathcal{S}||\mathcal{A}|b)$ & 分支因子$b$小的确定性环境 \\
        采样更新 & $O(1)$ & 大分支因子随机环境 \\
        \bottomrule
        \end{tabular}
    \end{table}

\subsubsection{决策时规划}
\paragraph{实时动态规划(RTDP)}
\begin{itemize}
    \item \textbf{特点}:
        \begin{itemize}
            \item 仅更新轨迹访问的状态
            \item 在跑道问题中更新次数比DP少50\%
        \end{itemize}
    \item \textbf{收敛条件}:
        \[
        \forall s \in \mathcal{S}_0, \lim_{k\to\infty} V_k(s) = V^*(s)
        \]
        其中$\mathcal{S}_0$为初始状态集
\end{itemize}

\subsubsection{蒙特卡洛树搜索(MCTS)}
\begin{algorithm}[H]
    \caption{MCTS核心流程}
    \begin{algorithmic}[1]
        \WHILE{计算资源未耗尽}
            \STATE \textbf{Selection}: 根据UCT选择节点
            \STATE \textbf{Expansion}: 扩展未探索动作
            \STATE \textbf{Simulation}: Rollout策略模拟
            \STATE \textbf{Backpropagation}: 回溯更新价值
        \ENDWHILE
        \RETURN 最优动作
    \end{algorithmic}
\end{algorithm}

\subsubsection{性能对比}
\begin{itemize}
\item \textbf{Dyna-Q}:
  \begin{itemize}
  \item 优点:模型学习稳定,适合中等规模问题。
  \item 缺点:内存占用较高(约50MB)。
  \item 平均运行时间:12.3秒。
  \end{itemize}

\item \textbf{RTDP}:
  \begin{itemize}
  \item 优点:实时性强,收敛最快(150次迭代)。
  \item 缺点:对初始策略敏感。
  \item 成功率:88\%。
  \end{itemize}

\item \textbf{MCTS}:
  \begin{itemize}
  \item 优点:无需环境模型,适合复杂决策。
  \item 缺点:计算资源需求高(120MB内存)。
  \item 适用场景:非确定性环境。
  \end{itemize}
\end{itemize}

\subsubsection{理论启示}
\begin{itemize}
    \item \textbf{三要素平衡}:
        \begin{itemize}
            \item 精度(期望更新)
            \item 效率(采样更新)
            \item 实时性(决策时规划)
        \end{itemize}
    \item \textbf{应用选择指南}:
       
        \begin{table}[H]  % 使用[H]强制定位
            \centering
            \caption{不同场景下的推荐算法}
            \begin{tabular}{ll}
            \toprule
            \textbf{场景} & \textbf{推荐算法} \\
            \midrule
            小规模确定性环境 & 优先级采样Dyna \\
            大规模连续空间 & RTDP \\
            离散动作决策优化 & MCTS \\
            \bottomrule
            \end{tabular}
        \end{table}

\end{itemize}

\section{近似逼近方法}
\subsection{近似逼近方法 (1)}

\subsubsection{大规模MDP的挑战}
\begin{itemize}
    \item 传统方法需维护查询表 ($V(s)$ 或 $Q(s,a)$),但面临:
    \begin{itemize}
        \item 状态/动作空间过大(如围棋 $10^{170}$ 状态)
        \item 连续状态/动作空间(如自动驾驶)
    \end{itemize}
    \item 解决方法:
    \begin{itemize}
        \item 离散化/分桶
        \item 参数化值函数估计
    \end{itemize}
\end{itemize}

\subsubsection{离散化方法}
\begin{itemize}
    \item \textbf{优点}: 直观、高效、适用于多数问题。
    \item \textbf{缺点}: 
    \begin{itemize}
        \item 简化价值函数表示(常数值假设)
        \item 维度灾难: $S = \mathbb{R}^n \Rightarrow \bar{S} = \{1, \dots, k\}^n$
    \end{itemize}
\end{itemize}

\subsubsection{参数化值函数近似}
\begin{itemize}
    \item 目标: 近似真实值函数
    \[
        V_\theta(s) \simeq V^\pi(s), \quad Q_\theta(s,a) \simeq Q^\pi(s,a)
    \]
    \item 参数 $\theta$ 通过强化学习更新,实现状态泛化。
    \item \textbf{函数形式}:
    \begin{itemize}
        \item 线性模型: $V_\theta(s) = \theta^T x(s)$
        \item 神经网络、决策树、傅立叶基底等
    \end{itemize}
\end{itemize}

\subsubsection{随机梯度下降 (SGD)}
\begin{itemize}
    \item 最小化均方误差:
    \[
        J(\theta) = \mathbb{E}_\pi \left[ \frac{1}{2} (V^\pi(s) - V_\theta(s))^2 \right]
    \]
    \item 梯度更新:
    \[
        \theta \leftarrow \theta + \alpha (V^\pi(s) - V_\theta(s)) \nabla_\theta V_\theta(s)
    \]
\end{itemize}

\subsubsection{算法实现}
\begin{itemize}
    \item \textbf{蒙特卡洛}:
    \[
        \theta \leftarrow \theta + \alpha (G_t - V_\theta(s)) x(s_t)
    \]
    \item \textbf{时序差分 (TD)}:
    \[
        \theta \leftarrow \theta + \alpha (r_{t+1} + \gamma V_\theta(s_{t+1}) - V_\theta(s)) x(s_t)
    \]
    \item \textbf{状态-动作值近似}:
    \[
        Q_\theta(s,a) = \theta^T x(s,a), \quad \theta \leftarrow \theta + \alpha (r_{t+1} + \gamma Q_\theta(s_{t+1},a_{t+1}) - Q_\theta(s,a)) x(s,a)
    \]
\end{itemize}

\subsubsection{关键结论}
\begin{itemize}
    \item 参数化方法优于离散化,适合非稳态、非独立同分布数据。
    \item 线性模型在蒙特卡洛和TD中可收敛至全局最优。
    \item TD目标含$\theta$但无需计算其梯度(因目标视为固定采样)。
\end{itemize}

\subsection{近似逼近方法 (2)}

\subsubsection{案例分析:Deep Q-Network (DQN)}
\begin{itemize}
    \item \textbf{核心思想}:使用深度神经网络表示Q函数,解决高维状态空间问题。
    \item \textbf{损失函数}:
    \[
    L_t(\theta_i) = \mathbb{E}_{(s,a,r,s') \sim U(D)} \left[ \left( r + \gamma \max_{a'} Q(s',a'; \theta_i^-) - Q(s,a; \theta_i) \right)^2 \right]
    \]
    \item \textbf{关键技术}:
    \begin{itemize}
        \item 目标网络($\theta_i^-$)与在线网络($\theta_i$)分离,稳定训练。
        \item 经验回放(Replay Buffer $D$),避免数据相关性。
    \end{itemize}
\end{itemize}

\subsubsection{策略梯度方法}
\paragraph{参数化策略}
\begin{itemize}
    \item 随机策略:$\pi_\theta(a|s) = P(a|s; \theta)$
    \item 确定性策略:$a = \pi_\theta(s)$
\end{itemize}

\paragraph{策略梯度定理}
\begin{itemize}
    \item 核心公式:
    \[
    \frac{\partial J(\theta)}{\partial \theta} = \mathbb{E}_{\pi_\theta} \left[ \frac{\partial \log \pi_\theta(a|s)}{\partial \theta} Q^{\pi}(s,a) \right]
    \]
    \item 适用于多步MDP,通过长期价值函数$Q^{\pi}(s,a)$替代瞬时奖励。
\end{itemize}

\paragraph{蒙特卡洛策略梯度(REINFORCE)}
\begin{itemize}
    \item 更新规则:
    \[
    \Delta \theta_t = \alpha \frac{\partial \log \pi_\theta(a_t|s_t)}{\partial \theta} G_t
    \]
    \item 优点:无偏估计;缺点:高方差、低数据效率。
    \item 伪代码:
    \begin{lstlisting}
initialize (*$\theta$*)
for each episode {s1,a1,r2,...,sT} ~ (*$\pi_\theta$*) do
    for t=1 to T-1 do
        (*$\theta \leftarrow \theta + \alpha \nabla_\theta \log \pi_\theta(a_t|s_t) G_t$*)
    end for
end for
    \end{lstlisting}
\end{itemize}

\paragraph{Softmax随机策略}
\begin{itemize}
    \item 策略定义:
    \[
    \pi_\theta(a|s) = \frac{e^{f_\theta(s,a)}}{\sum_{a'} e^{f_\theta(s,a')}}
    \]
    \item 梯度计算(线性参数化时):
    \[
    \frac{\partial \log \pi_\theta(a|s)}{\partial \theta} = x(s,a) - \mathbb{E}_{a' \sim \pi_\theta}[x(s,a')]
    \]
\end{itemize}

\subsubsection{Actor-Critic框架}
\paragraph{核心思想}
\begin{itemize}
    \item \textbf{Actor}(策略$\pi_\theta$):选择动作,通过策略梯度更新。
    \item \textbf{Critic}(值函数$Q_\Phi$):评估动作价值,指导Actor更新。
    \item 目标函数:
    \[
    J(\theta) = \mathbb{E}_{\pi_\theta} \left[ \pi_\theta(a|s) Q_\Phi(s,a) \right]
    \]
\end{itemize}

\paragraph{Advantage Actor-Critic (A2C)}
\begin{itemize}
    \item 优势函数:
    \[
    A^{\pi}(s,a) = Q^{\pi}(s,a) - V^{\pi}(s)
    \]
    \item 降低方差,提升训练稳定性:
    \[
    \frac{\partial J(\theta)}{\partial \theta} = \mathbb{E}_{\pi_\theta} \left[ \frac{\partial \log \pi_\theta(a|s)}{\partial \theta} A^{\pi}(s,a) \right]
    \]
\end{itemize}

\subsubsection{总结}
\begin{itemize}
    \item 策略梯度方法直接优化策略参数,适合连续动作空间和高维问题。
    \item Actor-Critic结合值函数与策略梯度,平衡偏差与方差。
    \item DQN与策略梯度是深度强化学习的核心基础,分别适用于离散和连续控制问题。
\end{itemize}

\section{深度强化学习价值方法}

\subsection{深度Q网络(DQN)}
\begin{itemize}
    \item \textbf{关键技术}:
    \begin{itemize}
        \item 经验回放:存储转移样本 \((s_t, a_t, r_t, s_{t+1})\) 到缓冲池,均匀采样以打破相关性。
        \item 目标网络:使用独立网络 \(Q_{\theta^-}\) 计算目标值,每隔\(C\)步同步参数。
    \end{itemize}
    \item \textbf{损失函数}:
    \[
    L(\theta) = \mathbb{E}_{(s,a,r,s') \sim D} \left[ \left( r + \gamma \max_{a'} Q_{\theta^-}(s', a') - Q_\theta(s, a) \right)^2 \right]
    \]
    \item \textbf{优点}:解决高维状态空间问题;\textbf{缺点}:Q值过估计。
\end{itemize}

\subsection{Double DQN}
\begin{itemize}
    \item \textbf{核心改进}:解耦动作选择与价值评估:
    \[
    y_t = r_t + \gamma Q_{\theta^-}(s_{t+1}, \arg\max_{a'} Q_\theta(s_{t+1}, a'))
    \]
    \item \textbf{解决过估计}:通过分离目标网络和策略网络,减少\(\max\)操作带来的偏差。
    \item \textbf{数学依据}:对于随机变量\(X_1, X_2\),有
    \[
    \mathbb{E}[\max(X_1, X_2)] \geq \max(\mathbb{E}[X_1], \mathbb{E}[X_2])
    \]
\end{itemize}

\subsection{Dueling DQN}
\begin{itemize}
    \item \textbf{网络结构}:将Q值分解为状态值\(V(s)\)和优势函数\(A(s,a)\):
    \[
    Q(s,a; \theta, \alpha, \beta) = V(s; \theta, \beta) + \left( A(s,a; \theta, \alpha) - \frac{1}{|A|}\sum_{a'} A(s,a'; \theta, \alpha) \right)
    \]
    \item \textbf{优势}:
    \begin{itemize}
        \item 更高效学习状态价值,尤其在动作影响较小时。
        \item 减少冗余学习,提升泛化能力。
    \end{itemize}
\end{itemize}

\subsection{Rainbow算法}
\begin{itemize}
    \item \textbf{集成技术}:结合DQN、Double DQN、优先经验回放、Dueling DQN、分布式DQN、NoisyNet等。
    \item \textbf{性能}:在Atari游戏中达到超越人类的表现。
\end{itemize}

\subsection{关键挑战与解决方案}
\begin{table}[H] % 大写H强制定位
\centering
\caption{深度强化学习挑战与解决方案}
\begin{tabular}{ll}
\toprule
\textbf{挑战} & \textbf{解决方案} \\
\midrule
高维状态空间 & 使用CNN提取特征 \\
数据相关性 & 经验回放缓冲池 \\
Q值过估计 & Double DQN \\
训练不稳定 & 目标网络 \\
稀疏奖励 & Dueling架构 \\
\bottomrule
\end{tabular}
\end{table}

\subsection{总结}
\begin{itemize}
    \item DQN系列算法通过神经网络近似Q函数,解决了传统强化学习在高维空间的局限性。
    \item 核心创新:经验回放、目标网络、值函数分解(Dueling)、解耦评估(Double DQN)。
    \item 未来方向:更高效的探索策略、多任务学习、理论收敛性分析。
\end{itemize}

\section{深度强化学习策略方法}
\subsection{深度强化学习分类}
\begin{itemize}
    \item \textbf{基于价值的方法}:DQN系列(Double DQN, Dueling DQN)
    \item \textbf{基于随机策略的方法}:
    \begin{itemize}
        \item REINFORCE:$\nabla_\theta J(\theta) = \mathbb{E}[\nabla_\theta \log \pi_\theta(a|s) G_t]$
        \item A3C:异步优势Actor-Critic
        \item PPO/TRPO:信任域策略优化
    \end{itemize}
    \item \textbf{基于确定性策略的方法}:
    \begin{itemize}
        \item DPG:$\nabla_\theta J = \mathbb{E}[\nabla_\theta \pi_\theta(s) \nabla_a Q^w(s,a)|_{a=\pi_\theta(s)}]$
        \item DDPG:DPG + 经验回放 + 目标网络
    \end{itemize}
\end{itemize}

\subsection{核心算法原理}
\subsubsection{策略梯度定理}
\[
\frac{\partial J(\theta)}{\partial \theta} = \mathbb{E}_{\pi_\theta}\left[\frac{\partial \log \pi_\theta(a|s)}{\partial \theta} Q^{\pi_\theta}(s,a)\right]
\]
\begin{itemize}
    \item 适用于随机策略(Softmax形式):
    \[
    \pi_\theta(a|s) = \frac{e^{f_\theta(s,a)}}{\sum_{a'} e^{f_\theta(s,a')}}
    \]
    \item 梯度计算:
    \[
    \nabla_\theta \log \pi_\theta(a|s) = \nabla_\theta f_\theta(s,a) - \mathbb{E}_{a'}[\nabla_\theta f_\theta(s,a')]
    \]
\end{itemize}

\subsubsection{A3C (异步优势Actor-Critic)}
\begin{itemize}
    \item \textbf{核心思想}:
    \begin{itemize}
        \item 并行多个Worker与环境交互
        \item 使用优势函数降低方差:$A(s,a) = \sum_{i=0}^{k-1} \gamma^i r_{t+i} + \gamma^k V(s_{t+k}) - V(s_t)$
    \end{itemize}
    \item \textbf{伪代码}:
    \begin{algorithmic}[1]
        \STATE 全局共享参数 $\theta$, $\theta_v$
        \FOR{每个Worker线程}
        \STATE 同步线程参数 $\theta' \gets \theta$, $\theta_v' \gets \theta_v$
        \STATE 采集轨迹数据 $\tau = \{s_t,a_t,r_t\}$
        \STATE 计算优势 $A(s_t,a_t) = R_t - V(s_t;\theta_v')$
        \STATE 累计梯度 $d\theta \gets \sum \nabla_{\theta'} \log \pi(a_t|s_t)A(s_t,a_t)$
        \STATE 异步更新全局参数
        \ENDFOR
    \end{algorithmic}
\end{itemize}

\subsubsection{DDPG (深度确定性策略梯度)}
\begin{itemize}
    \item \textbf{关键技术}:
    \begin{itemize}
        \item 确定性策略:$a = \pi_\theta(s)$
        \item Critic网络:最小化TD误差 $L(w) = \mathbb{E}[(Q^w(s,a) - Q^\pi(s,a))^2]$
        \item 目标网络软更新:$\theta' \gets \tau \theta + (1-\tau)\theta'$
    \end{itemize}
    \item \textbf{策略梯度}:
    \[
    \nabla_\theta J = \mathbb{E}_{s\sim \rho^\pi}[\nabla_\theta \pi_\theta(s) \nabla_a Q^w(s,a)|_{a=\pi_\theta(s)}]
    \]
\end{itemize}

\subsection{对比实验}
\begin{table}[H] % 大写H强制定位
\centering
\caption{Atari游戏性能对比}
\begin{tabular}{lcc}
\toprule
方法 & 训练时间 & 人类标准化得分 \\ 
\midrule
DQN & 8天 (GPU) & 121.9\% \\
A3C (CPU) & 1天 & 344.1\% \\
DDPG & 4天 (GPU) & 优于DPG 200\% \\
\bottomrule
\end{tabular}
\end{table}

\subsection{关键结论}
\begin{itemize}
    \item \textbf{随机策略}:适合离散动作空间,但高方差
    \item \textbf{确定性策略}:适合连续控制,需配合探索噪声
    \item \textbf{并行训练}:A3C通过异步更新加速收敛
    \item \textbf{稳定性技巧}:目标网络、经验回放对DDPG至关重要
\end{itemize}


\section{写在最后}
\subsection{发布地址}
\begin{itemize}
    \item Github: 
    \url{https://github.com}

\end{itemize}